\documentclass[a4paper,11pt]{article}

% Use UTF-8 for input encoding
\usepackage[utf8]{inputenc}

% Use T1 font encoding
\usepackage[T1]{fontenc}

% Hyphenation for supported languages
\usepackage[english,ngerman]{babel}

% Floats
\usepackage{float}

% Advanced tables
\usepackage{array}

% Mathematics
\usepackage{amsmath}
\usepackage{amssymb}

% TikZ pictures
\usepackage{tikz}

% Source code listing
\usepackage{listings}

% Conditional branching
\usepackage{ifthen}

% Multilanguage support
% (see http://en.wikibooks.org/wiki/LaTeX/Internationalization#Multilingual_versions)
\def\addlocale#1#2{
\ifthenelse{ \equal{\locale}{#1} }{
  \selectlanguage{#2}
  \expandafter\def\csname#1\endcsname ##1{##1}
  }{
  \expandafter\def\csname#1\endcsname ##1{}
  }
}

% Define supported languages
\addlocale{en}{english}
\addlocale{de}{ngerman}

% Support for hyperlinks
\usepackage[
  hidelinks,
  pdftitle={WWU Beamervorlage},
  pdfsubject={},
  pdfauthor={Dominik Lekse},
  pdfkeywords={}
]{hyperref}

% Styles
% Use smaller margins (set the margins to one inch)
\usepackage[margin=2.54cm]{geometry}

% Indention
\setlength{\parindent}{0cm}

% List formatting
\renewcommand{\labelitemi}{\raisebox{0.75pt}{\scriptsize$\blacksquare$}}

% Define ERCIS colors (see official corporate design manual)
\definecolor{ercisblack}{RGB}{0,0,0}
\definecolor{ercisgrey}{RGB}{94,94,93}
\definecolor{ercisred}{RGB}{133,35,57}
\definecolor{ercislightred}{RGB}{200,156,166}
\definecolor{ercisblue}{RGB}{135,151,163}
\definecolor{ercisclaim}{named}{ercislightred}

% Additional ERCIS colors
\definecolor{ercisdarkblue}{RGB}{67,92,139}
\definecolor{erciscyan}{RGB}{0,156,179}
\definecolor{ercisorange}{RGB}{231,124,18}
\definecolor{ercisgreen}{RGB}{135,191,42}

% Define WWU colors (see official Corporate Design manual, p. 21)
\definecolor{pantoneblack7}{RGB}{62,62,59}


% Document variables
\newcommand{\varstylename}{ercisbeamer}
\newcommand{\varstylefile}{\varstylename.sty}

% Meta information
\title{\en{ERCIS beamer template}\de{ERCIS Beamervorlage}}
\author{Dominik Lekse\\
lekse@lugmedia.de}
\date{Stand: 11.05.2012}

\begin{document}
% Select document language
\en{\selectlanguage{english}}
\de{\selectlanguage{ngerman}}

% Title
\maketitle

% Table of contents
\renewcommand{\contentsname}{\en{Contents}\de{Inhalt}}
\tableofcontents
\pagebreak

\section{\en{Prerequisites}\de{Voraussetzungen}}
\en{Prior to the use of the template, the following requirements must be satisfied to ensure proper rendering of presentations.}
\de{Für eine einwandfreie Darstellung der Präsentationen müssen vor der Verwendung der Vorlage die folgenden Voraussetzungen erfüllt sein.}

\begin{itemize}
  \item
    \en{Install the typeface \emph{FF Meta}, which is available for members of the University of Münster in the \href{https://www.uni-muenster.de/marketing/corporatedesign/schriften.html}{corporate design manual}.}
    \de{Installieren Sie die Schriftart \emph{FF Meta}, die Angehörigen der Universität Münster im \href{https://www.uni-muenster.de/marketing/corporatedesign/schriften.html}{Corporate Design Manual} zur Verfügung steht.}
\end{itemize}

\section{\en{Notes on Usage}\de{Hinweise zur Verwendung}}
\en{To create a presentation using the template, the following issues have to be considered.}
\de{Bei der Erstellung einer neuen Präsentation mit der Vorlage, müssen die folgenden Hinweise beachtet werden.}

\begin{itemize}
  \item
    \en{Copy the style \emph{\varstylefile} in the folder which contains the source files of your presentation.}
    \de{Kopieren Sie den Stil \emph{\varstylefile} in den Ordner, der die Quelldateien ihrer Präsentation enthält.}
  \item
    \en{Add the line \emph{\textbackslash usepackage[...]\{\varstylename\}} including the appropriate options to the preamble of your presentation.}
    \de{Fügen Sie \emph{\textbackslash usepackage[...]\{\varstylename\}} mit den entsprechenden Optionen der Präambel Ihrer Präsentation hinzu.}
  \item
    \en{Create the sub folders \emph{beamer-cache} and \emph{handout-cache} inside your presentation folder. These folders are used by the template to cache embedded graphics depending on the current mode (\emph{beamer} or \emph{handout}). For more information on the caching mechanism, refer to section~\ref{sec:features-caching}.}
    \de{Erstellen Sie im Präsentationordner die Unterordner \emph{beamer-cache} sowie \emph{handout-cache}. In diesen Ordnern werden abhänig vom aktuellen Modus (\emph{beamer} oder \emph{Handout}) für das Präsentationlayout benötigte bzw. Ihre eigenen TikZ-Grafiken zwischengespeichert. Weitere Informationen zum Caching-Mechanismus können Sie Abschnitt~\ref{sec:features-caching} entnehmen.}
  \item
    \en{Compile the presentation using the command line argument \emph{-shell-escape}. This is required by the \LaTeX\ compiler to invoke the sub processes for caching individual graphics.}
    \de{Kompilieren Sie Ihre Präsentation mit dem Argument \emph{-shell-escape}. Dies erlaubt dem \LaTeX Kompiler Unterprozesse für die Zwischenspeicherung einzelner Grafiken zu starten}
\end{itemize}

% TODO include bootstrap presentation

\section{\en{Features}\de{Funktionen}}

\en{An overview of the features provides by the template is given in section~\ref{sec:features-overview}. In addition, the distribution of the template comprises the source \emph{presentation.tex}, which contains the use of elements commonly used in presentation.}
\de{Eine Übersicht der Funktionen der Vorlage ist in Abschnitt~\ref{sec:features-overview} dargestellt. Zusätzlich befindet sich im Vorlagenpaket das Dokument \emph{presentation.tex}, im welchem exemplarisch die Verwendung von häufig benötigten Präsentationselementen veranschaulicht.}

\subsection{\en{Overview}\de{Übersicht}}
\label{sec:features-overview}

\begin{itemize}
  \item
    \en{Support for all elements of the beamer package}
    \de{Unterstützung aller Gestaltungselemente des Beamer-Pakets}
  \item
    \en{Multiple display formats}
    \de{Mehrere Anzeigeformate}
  \item
    \en{Caching of TikZ pictures}
    \de{Zwischenspeichern von TikZ Grafiken}
\end{itemize}

\subsection{\en{Display formats}\de{Anzeigeformate}}
\label{sec:features-formats}

\en{The template supports three common display formats, which are summarized in the following table~\ref{tab:features-formates}. To define the display format of a presentation, the corresponding option have to be added to the option list of the \emph{\textbackslash usepackage[...]\{\varstylename\}} command in its preamble.}
\de{Die Vorlage unterstützt drei verschiedene Anzeigeformate, die in der folgenden Tabelle~\ref{tab:features-formates} zusammengefasst sind. Falls kein Format explizit angegeben wird, werden Präsentationen im 4:3-Format erstellt.}

% Style to add white space at the top of a picture
\tikzset{
    cellpicture/.append style={
        execute at end picture={
            \path (current bounding box.south west) -- +([yshift=0.5em] current bounding box.north east);
        }
    }
}

\begin{table}[H]%
  \begin{center}%
    \renewcommand{\arraystretch}{1.25}
    \begin{tabular}{>{\centering} m{2cm} >{\centering} m{2cm} >{\centering} m{2cm} >{\centering} m{2cm} >{\centering} m{2cm}}
      \hline
      \textbf{\en{Preview}\de{Vorschau}} & \textbf{\en{Aspect ratio}\de{Format}} & \textbf{\en{Option}\de{Option}} & \textbf{\en{Width}\de{Breite}} & \textbf{\en{Height}\de{Höhe}} \tabularnewline
      \hline
      \tikz[cellpicture]{\path[draw] (0,0) rectangle (15mm, 11.25mm);} & 4:3 & - & 128 mm & 96 mm \tabularnewline
      \tikz{\path[draw] (0,0) rectangle (15mm, 9.375mm);} & 16:10 & wide10 & 128 mm & 80 mm \tabularnewline
      \tikz{\path[draw] (0,0) rectangle (15mm, 8.4375mm);} & 16:9 & wide9 & 128 mm & 72 mm \tabularnewline
      \hline
    \end{tabular}
    \caption{\en{Supported display formats}\de{Unterstützte Anzeigeformate}\label{tab:features-formates}}%
    \renewcommand{\arraystretch}{1}
  \end{center}%
\end{table}%

\subsection{Font selection}
% TODO describe fallback mechanism

\subsection{\en{Languages}\de{Sprachen}}
\en{}
Die Sprache der Präsentation wird über eine Option festgelegt und beeinflusst den Schriftzug des Claims auf der Titelseite und in der Seitenleiste einer Folie. Falls keine Option angegeben wird, ist \emph{deutsch} als Sprache voreingestellt.
\begin{table}[H]%
  \begin{center}%
    \renewcommand{\arraystretch}{1.25}
    \begin{tabular}{>{\centering} m{2cm} >{\centering} m{2cm}}
      \hline
      \textbf{Spache} & \textbf{Option} \tabularnewline
      \hline
      Deutsch & german \tabularnewline
      Englisch & english \tabularnewline
      \hline
    \end{tabular}
    \renewcommand{\arraystretch}{1}
  \end{center}%
\end{table}%

\subsection{\en{Colors}\de{Farben}}
\en{The template provides the following named colors for use inside a presentation. These colors colors comply with the ERCIS corporate design.}
\de{Die Vorlage stellt die folgenden benannten Farben für eine Verwendung in einer Präsentation zur Verfügung. Diese Farben entsprechen dem ERCIS Corporate Design.}

\begin{table}[H]%
  \begin{center}%
    \tikzstyle{colorsample}=[rectangle, minimum width=7.5mm, minimum height=7.5mm, use as bounding box, outer sep=2em]
    \renewcommand{\arraystretch}{1.25}
    \begin{tabular}{>{\centering} m{2cm} >{\centering} m{4cm} >{\centering} m{6cm}}
      \hline
      \textbf{\en{Color}\de{Farbe}} & \textbf{\en{Name}\de{Name}} & \textbf{\en{RGB / Hexadecimal}\de{RGB / Hexadezimal}} \tabularnewline
      \hline
      \tikz[cellpicture]{\node[colorsample, fill=ercisblack] {};} & ercisblack & 0 0 0 / \#000000 \tabularnewline
      \tikz{\node[colorsample, fill=ercisgrey] {};} & ercisgrey & 94 94 93 / \#5e5e5d \tabularnewline
      \hline
      \tikz[cellpicture]{\node[colorsample, fill=ercisred] {};} & ercisred & 133 35 57 / \#852339 \tabularnewline
      \tikz{\node[colorsample, fill=ercislightred] {};} & ercislightred & 200 156 166 / \#c89ca6 \tabularnewline
      \tikz{\node[colorsample, fill=ercisblue] {};} & ercisblue & 135 151 163 / \#8797a3 \tabularnewline
      \hline
      \tikz[cellpicture]{\node[colorsample, fill=ercisdarkblue] {};} & ercisdarkblue & 67 92 139 / \#435c8b \tabularnewline
      \tikz{\node[colorsample, fill=erciscyan] {};} & erciscyan & 0 156 179 / \#009cb3 \tabularnewline
      \tikz{\node[colorsample, fill=ercisorange] {};} & ercisorange & 231 124 18 / \#e77c12 \tabularnewline
      \tikz{\node[colorsample, fill=ercisgreen] {};} & ercisgreen & 135 191 42 / \#87bf2a \tabularnewline
      \hline
      \tikz[cellpicture]{\node[colorsample, fill=pantoneblack7] {};} & pantoneblack7 & TODO \tabularnewline
      \hline
    \end{tabular}
    \renewcommand{\arraystretch}{1}
  \end{center}%
\end{table}%

\subsection{\en{Caching of TikZ pictures}\de{Zwischenspeichern von TikZ Grafiken}}
\label{sec:features-caching}
\en{}
In der Voreinstellung werden alle in einer Präsentation eingebetteten TikZ-Grafiken gecached, d.h. sie werden einmalig in ein separates PDF-Dokument ausgelagert und bei einer erneuten Kompilierung nicht neu erstellt. Dadurch können Präsentationen, in denen viele TikZ-Grafiken eingebettet sind, schneller erstellt werden.

% TODO write more details on caching
% naming of pictures is required
% overlay pictures cannot be cached: try to avoid overlay pictures (even 2nd reason: the \LaTeX \ compiler needs two runs for an overlay picture to be position properly.)

% TODO create table for helper commands
\emph{\textbackslash rerenderAll} Erstellt alle TikZ-Grafiken neu und cached diese.

\emph{\textbackslash rerenderPicture} Erstellt die folgende TikZ-Grafik neu und cached diese.

\emph{\textbackslash nocachePicture} Verhindert das Caching der folgenden TikZ-Grafik.

% Force the rerendering of the subsequent picture
\providecommand{\rerenderPicture}{%
  \tikzset{external/remake next=true}%
}

% % Force the regeneration of pictures marked as cached once
% \providecommand{\rerenderAllCachedOnce}{%
%   \global\def\rerendercachedonceoption{\empty}%
% }

% % Force the regeneration of all pictures
% \providecommand{\rerenderAll}{%
%   \tikzset{external/force remake=true}%
% }

% % Explicitly enable caching of the subsequent picture
% \providecommand{\cachePicture}{%
%   \tikzset{external/export next=true}%
% }

% % Explicitly disable caching of the subsequent picture
% \providecommand{\nocachePicture}{%
%   \tikzset{external/export next=false}%
% }

% % Explicitly disable caching of all pictures
% \providecommand{\nocacheAll}{%
%   \tikzset{external/export=false}%
% }

% TODO more helper

% TODO tipp zum workflow für das erstellen von folien
% add nocachePicture in front of unfinished

% \subsubsection*{Deaktivieren des Caching von TikZ-Grafiken}
% Option \emph{nocache}

\section{\en{Display}\de{Darstellung}}

\subsection{Bleeding logo in Adobe Reader and Adobe Acrobat Professional}

gap between the boxes in the ERCIS logo is narrow
Viewing the presentation in Adobe Reader and Adobe Acrobat Professional
disable the option Edit > Preferences > Page Display > Enhance thin lines

\end{document}
